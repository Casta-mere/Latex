\documentclass[10pt,A4]{article}	

\usepackage[utf8]{inputenc}		
\usepackage{xstring, xifthen} %去掉了-
\renewcommand*\familydefault{\sfdefault} 
\usepackage[T1]{fontenc}
\usepackage{moresize}


% 图标库
\usepackage{fontawesome}
% use to vertically center content
% credits to: http://tex.stackexchange.com/questions/7219/how-to-vertically-center-two-images-next-to-each-other
\newcommand{\vcenteredinclude}[1]{\begingroup
\setbox0=\hbox{\includegraphics{#1}}%
\parbox{\wd0}{\box0}\endgroup}

% use to vertically center content
% credits to: http://tex.stackexchange.com/questions/7219/how-to-vertically-center-two-images-next-to-each-other
\newcommand*{\vcenteredhbox}[1]{\begingroup
\setbox0=\hbox{#1}\parbox{\wd0}{\box0}\endgroup}

% icon shortcut
\newcommand{\icon}[3] { 							
	\makebox(#2, #2){\textcolor{maincol}{\csname fa#1\endcsname}}
}	

% icon with text shortcut
\newcommand{\icontext}[4]{ 						
	\vcenteredhbox{\icon{#1}{#2}{#3}}  \hspace{2pt}  \parbox{0.7\mpwidth}{\textcolor{#4}{#3}}
}

% icon with website url
\newcommand{\iconhref}[5]{ 						
    \vcenteredhbox{\icon{#1}{#2}{#5}}  \hspace{2pt} \href{#4}{\textcolor{#5}{#3}}
}

% icon with email link
\newcommand{\iconemail}[5]{ 						
    \vcenteredhbox{\icon{#1}{#2}{#5}}  \hspace{2pt} \href{mailto:#4}{\textcolor{#5}{#3}}
}


% 分栏
\usepackage{paracol}

% 缩小页边距
\usepackage[a4paper]{geometry}
\geometry{top=1cm, bottom=1cm, left=1cm, right=1cm}

% 页眉页脚
\usepackage{fancyhdr}
\pagestyle{empty}

% 首行缩进
\setlength{\parindent}{0mm}

% 表格
\usepackage{array}
\newcolumntype{x}[1]{%
>{\raggedleft\hspace{0pt}}p{#1}}%

% 插入图片 (\includegraphics)
\usepackage{graphicx}

% 画图
\usepackage{tikz}				
\usetikzlibrary{shapes, backgrounds,mindmap, trees}

% 定义颜色宏
\usepackage{transparent}
\usepackage{color}
\definecolor{maincol}{RGB}{ 225, 0, 0 }
\definecolor{darkcol}{RGB}{ 70, 70, 70 }
\definecolor{lightcol}{RGB}{245,245,245}


% 超链接
\usepackage[hidelinks]{hyperref}

% 中文
\usepackage{xeCJK}

\usepackage{fontspec}

\usepackage[default]{raleway}
\setCJKmainfont{Microsoft YaHei}
\setmainfont{Raleway}
\renewcommand{\CJKglue}{\hskip 1pt}

%----------------------------------------------------------------------------------------
%   自定义命令
%----------------------------------------------------------------------------------------

\newcommand{\mpwidth}{\linewidth-\fboxsep-\fboxsep} % 页宽

\newcommand{\cvtext}[1] {
	\begin{tabular*}{1\mpwidth}{p{0.98\mpwidth}}
		\parbox{1\mpwidth}{#1}
	\end{tabular*}
}

\newcommand{\cvsection}[1] {
	\vspace{14pt}
	\cvtext{
        % 使用 group 在内部修改间距等
        \begingroup
        \renewcommand{\CJKglue}{\hskip 2.5pt}
		\textbf{\LARGE{\textcolor{darkcol}{\uppercase{#1}}}}\\[-4pt]
		\textcolor{maincol}{ \rule{0.1\textwidth}{2pt} } \\
        \endgroup
        }
}

\newcommand{\cvskill}[3] {
	\begin{tabular*}{1\mpwidth}{p{0.72\mpwidth}  r}
 		\textcolor{black}{\textbf{#1}} & \textcolor{maincol}{#2}\\
	\end{tabular*}%
	
	\hspace{4pt}
	\begin{tikzpicture}[scale=1,rounded corners=2pt,very thin]
		\fill [lightcol] (0,0) rectangle (1\mpwidth, 0.15);
		\fill [maincol] (0,0) rectangle (#3\mpwidth, 0.15);
  	\end{tikzpicture}%
}

\newcommand{\cvcertificate}[2] {
	\begin{tabular*}{1\mpwidth}{p{0.8\mpwidth}    r}
 		\textcolor{black}{\textbf{#1}} & \textcolor{black}{\textbf{#2}}\\
	\end{tabular*}%
	
}

\newcommand{\cvlist}[1] {
	\begin{itemize}{#1}\end{itemize}

    \vspace{8pt}

    }

\newcommand{\eduevent}[6] {
	
	\parbox{\mpwidth}{
		\begin{tabular*}{1\mpwidth}{p{0.69\mpwidth}  r}
	 		{\large \textcolor{black}{\textbf{#2}}}& \colorbox{maincol}{\makebox[0.25\mpwidth]{\textcolor{white}{#1}}} \\[7pt]
			\textcolor{darkcol}{\textbf{#3}} & \\[7pt]
            \textcolor{darkcol}{\textbf{#4}} & \\[7pt]
            #5 & \\
		\end{tabular*}\\
	}
    {#6}
}

\newcommand{\cvproject}[3] {

	\parbox{\mpwidth}{
		\begin{tabular*}{1\mpwidth}{p{0.65\mpwidth}  r}
            {\large \textcolor{black}{\textbf{#2}}} & \colorbox{maincol}{\makebox[0.33\mpwidth]{\textcolor{white}{#1}}} \\
		\end{tabular*}\\
	}
    \begingroup
        \renewcommand{\CJKglue}{\hskip 1.5pt}
        #3
    \endgroup
}

%----------------------------------------------------------------------------------------
%	main
%----------------------------------------------------------------------------------------
\begin{document}

\columnratio{0.7} % 左栏占比70%
\setlength{\columnsep}{2.2em} % 左右栏间距
\setlength{\columnseprule}{4pt} % 左右栏间距宽度
\colseprulecolor{lightcol} % 左右栏间距颜色

% 分栏
\begin{paracol}{2}

% 左栏
\begin{leftcolumn}

\cvsection{教育经历}
\eduevent
	{2020-09\quad 至今}
	{浙江理工大学}
	{计算机科学与技术\enspace (全英文)\enspace |\enspace 本科}
	{GPA\enspace 3.76/5.00\enspace (前15$\%$)}
    {相关课程}
    {   

        \cvlist{
            \item C程序设计 {\hfill (96) \qquad \qquad \enspace \enspace }
            \item Python高阶程序设计 {\hfill (92) \qquad \qquad \enspace \enspace }
            \item 面向对象程序设计 {\hfill (93) \qquad \qquad \enspace \enspace }
            \item 数据结构与算法 {\hfill (96) \qquad \qquad \enspace \enspace }
            \item 计算机网络 {\hfill (95) \qquad \qquad \enspace \enspace }  
        }
    }

\vspace{-10pt}

\vfill\null

\cvsection{项目经历}
\cvproject
    {2022-03\enspace 至\enspace 2022-08}
    {基于深度学习的商品识别系统}
    {\cvlist{
            \item 针对在无人货架中的应用,对常见的20余种商品,拍摄了共5422张不同角度,不同组合的图片,以及对应的标注文件作为数据集
            \item 使用Python语言,\textbf{PyTorch}框架,\textbf{YOLOv5}算法。训练过程中对训练数据集使用了Mosaic拼接、图片覆盖混入、随机翻转等数据增强方式。通过200轮的迭代,最终\textbf{mAP 0.5:0.95}达到\textbf{0.82}以上
            \item 最终部署至RKNN边缘智能计算平台以及硬件设备中
        }
    }

\cvproject
    {2022-09\enspace 至\enspace 2022-12}
    {智能知识侦察助手}
    {\cvlist{
            \item 爬取豆瓣图书Top200的书籍,编写对图书信息侦查的问卷生成程序,并且以网页形式呈现出来,实现问卷填写,数据分析,可视化展示等
            \item 本人主要负责\textbf{后端开发}与\textbf{服务器部署}。后端使用Python语言,应用面向对象程序设计的思想抽象出书籍、问题、数据库连接、控制等多个类进行操作。最终部署至Linux服务器。网页链接:\href{http://castamerego.com/}{\textbf{\emph{castamerego.com}}}
            \item 三人小组使用Github,拥有\textbf{超过100条}commit记录,并有完整的过程记录,具体内容可访问\href{https://github.com/Castamere/Read-Book}{\textbf{\emph{github.com/Castamere/Read-Book}}}查看
        }
    }

\cvproject
    {2022-04\enspace 至\enspace 2022-05}
    {大众点评爬虫}
    {\cvlist{
            \item 使用\textbf{selenium}和 \textbf{requests}库爬取数据。针对大众点评的\textbf{css-svg映射}爬取进行逆向
            \item 使用id,class,xpath等多种方式对元素定位,用\textbf{beautifulSoup4}和re库清洗数据,并通过\textbf{Pandas},PyMysql等库进行数据存储
        }
    }

\vspace{-10pt}

\vfill\null

\cvsection{获奖经历}

\cvlist{
		\item 中国大学生计算机设计大赛\quad \textbf{国家级三等奖}
        \item 中国大学生英语阅读大赛\quad \textbf{校级特等奖}
        \item 大学生电子商务竞赛\quad  省级三等奖
        \item 互联网+创新创业大赛\quad  校级二等奖
        \item 浙江省\quad \textbf{省政府奖学金} 
        \item 校二等、三等奖学金、优秀学生干部、社会工作奖学金
        
}

\vfill\null

% \cvsection{自我评价}

% \vfill\null


\end{leftcolumn}

% 右栏
\begin{rightcolumn}

\includegraphics[width=\linewidth]{蓝底渐变.png}

\vfill\null

\cvsection{王旭刚}

\icontext{MobilePhone}{12}{138\,-\,3423\,-\,0484}{black}\\[6pt]
\icontext{Qq}{12}{2287245796}{black}\\[6pt]
\iconemail{Envelope}{12}{castamerego@gmail.com}{castamerego@gmail.com}{black}\\[6pt]
\iconhref{Github}{12}{github.com/Casta-mere}{https://github.com/Casta-mere}{black}\\[6pt]
\icontext{MapMarker}{12}{浙江 \enspace 杭州}{black}
% \iconhref{Globe}{12}{castamerego.com}{http://castamerego.com/}{black}\\[6pt]

\vfill\null

\cvsection{专业技能}

\cvskill{Python} {3+ yrs} {1} \\[-2pt]

\cvskill{C++} {3+ yrs} {0.92} \\[-2pt]

\cvskill{Linux} {2+ yrs} {0.59} \\[-2pt]

\cvskill{GIT} {3+ yrs} {0.68} \\[-2pt]

\cvskill{Mysql} {2+ yrs} {0.68} \\[-2pt]

\cvskill{\LaTeX } {2+ yrs} {0.82} \\[-2pt]

\cvskill{HTML/CSS/Javascript} {掌握} {0.46} \\[-2pt]

\cvskill{Ms Of\mbox{}f\mbox{}ice} {精通} {1} \\[-2pt]

\vfill\null

\cvsection{资格证书}

\cvcertificate{CET-4}{\texttt{571}}

\cvcertificate{CET-6}{\texttt{508}}

\vfill\null

\end{rightcolumn}
\end{paracol}

\end{document}
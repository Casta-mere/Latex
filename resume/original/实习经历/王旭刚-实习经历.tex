\documentclass[10pt,A4]{article}	

\usepackage[utf8]{inputenc}		
\usepackage{xstring, xifthen} %去掉了-
\renewcommand*\familydefault{\sfdefault} 
\usepackage[T1]{fontenc}
\usepackage{moresize}


% 图标库
\usepackage{fontawesome}
% use to vertically center content
\newcommand{\vcenteredinclude}[1]{\begingroup
\setbox0=\hbox{\includegraphics{#1}}%
\parbox{\wd0}{\box0}\endgroup}

% use to vertically center content
\newcommand*{\vcenteredhbox}[1]{\begingroup
\setbox0=\hbox{#1}\parbox{\wd0}{\box0}\endgroup}

% icon shortcut
\newcommand{\icon}[3] { 							
	\makebox(#2, #2){\textcolor{maincol}{\csname fa#1\endcsname}}
}	

% icon with text shortcut
\newcommand{\icontext}[4]{ 						
	\vcenteredhbox{\hspace{1pt} \icon{#1}{#2}{#3}}  \hspace{1pt}  \parbox{0.7\mpwidth}{\textcolor{#4}{#3}}
}

% icon with website url
\newcommand{\iconhref}[5]{ 						
    \vcenteredhbox{\hspace{1pt} \icon{#1}{#2}{#5}}  \hspace{1pt} \href{#4}{\textcolor{#5}{#3}}
}

% icon with email link
\newcommand{\iconemail}[5]{ 						
    \vcenteredhbox{\hspace{1pt} \icon{#1}{#2}{#5}}  \hspace{1pt} \href{mailto:#4}{\textcolor{#5}{#3}}
}


% 分栏
\usepackage{paracol}

% 缩小页边距
\usepackage[a4paper]{geometry}
\geometry{top=1cm, bottom=1cm, left=1cm, right=1cm}

% 页眉页脚
\usepackage{fancyhdr}
\pagestyle{empty}

% 首行缩进
\setlength{\parindent}{0mm}

% 表格
\usepackage{array}
\newcolumntype{x}[1]{%
>{\raggedleft\hspace{0pt}}p{#1}}%

% 插入图片 (\includegraphics)
\usepackage{graphicx}

% 画图
\usepackage{tikz}				
\usetikzlibrary{shapes, backgrounds,mindmap, trees}

% 定义颜色宏
\usepackage{transparent}
\usepackage{color}
\definecolor{maincol}{RGB}{ 225, 0, 0 }
\definecolor{darkcol}{RGB}{ 70, 70, 70 }
\definecolor{lightcol}{RGB}{225,225,225}


% 超链接
\usepackage[hidelinks]{hyperref}

% 中文
\usepackage{xeCJK}

\usepackage{fontspec}
\usepackage{sourcecodepro}

\usepackage[default]{raleway}
\setCJKmainfont{Microsoft YaHei}
\setmainfont{Raleway}
\renewcommand{\CJKglue}{\hskip 1.5pt}

\usepackage{enumitem}

\setitemize[1]{itemsep=3pt,partopsep=0pt,parsep=\parskip,topsep=2pt}
\usepackage{multicol}
%----------------------------------------------------------------------------------------
%   自定义命令
%----------------------------------------------------------------------------------------

\newcommand{\mpwidth}{\linewidth-\fboxsep-\fboxsep} % 页宽

\newcommand{\cvtext}[1] {
	\begin{tabular*}{1\mpwidth}{p{0.98\mpwidth}}
		\parbox{1\mpwidth}{#1}
	\end{tabular*}
}

\newcommand{\mysection}[1] {
	
    \vspace{12pt}

	\cvtext{
        % 使用 group 在内部修改间距等
        \begingroup
        \renewcommand{\CJKglue}{\hskip 2.5pt}
		\textbf{\LARGE{\textcolor{darkcol}{\uppercase{#1}}}}\\[-4pt]
		\textcolor{maincol}{ \rule{0.143\textwidth}{2pt} } \\[-6pt]
        \endgroup
        }
}

\newcommand{\name}[1] {
	
    \vspace{12pt}

	\cvtext{
        % 使用 group 在内部修改间距等
        \begingroup
        \renewcommand{\CJKglue}{\hskip 2.5pt}
		\textbf{\LARGE{\textcolor{darkcol}{\uppercase{#1}}}}\\[-4pt]
		\textcolor{maincol}{ \rule{0.105\textwidth}{2pt} } \\[-6pt]
        \endgroup
        }
}

\newcommand{\skill}[3] {
	\begin{tabular*}{1\mpwidth}{p{0.62\mpwidth}  r}
 		\textcolor{black}{\textbf{#1}} & \textcolor{maincol}{\texttt{\textbf{#2}}}\\
	\end{tabular*}%
	
	\hspace{4pt}
	\begin{tikzpicture}[scale=1,rounded corners=2pt,very thin]
		\fill [lightcol] (0,0) rectangle (1\mpwidth, 0.15);
		\fill [maincol] (0,0) rectangle (#3\mpwidth, 0.15);
  	\end{tikzpicture}%
}

\newcommand{\certification}[2] {
	\begin{tabular*}{1\mpwidth}{p{0.8\mpwidth}    r}
 		\textcolor{black}{\textbf{#1}} & \textcolor{black}{\textbf{#2}}\\
	\end{tabular*}%
	
}

\newcommand{\cvlist}[1] {


	\begin{itemize}
        {#1}
    \end{itemize}

    \vspace{3pt}

    }

\newcommand{\education}[6] {
	
	\parbox{\mpwidth}{
		\begin{tabular*}{1\mpwidth}{p{0.69\mpwidth}  r}
	 		{\large \textcolor{black}{\textbf{#2}}} & \colorbox{maincol}{\makebox[0.25\mpwidth]{\textcolor{white}{\textbf{#1}}}} \\[3pt]
			\textcolor{darkcol}{\textbf{#3}} & \\[3pt]
            \textcolor{darkcol}{\textbf{#4}} & \\[3pt]
            #5 & \\
		\end{tabular*}\\
	}
    {#6}
}

\newcommand{\project}[3] {
    \vspace{2pt}

	\parbox{\mpwidth}{
		\begin{tabular*}{1\mpwidth}{p{0.65\mpwidth}  r}
            {\Large \textcolor{black}{\textbf{#2}}} & \colorbox{maincol}{\makebox[0.33\mpwidth]{\textcolor{white}{\textbf{#1}}}} \\
		\end{tabular*}\\
	}
    \begingroup
        \setlength{\baselineskip}{15pt}
        \renewcommand{\CJKglue}{\hskip 1.5pt}
        #3
    \endgroup
}

\newcommand{\intership}[4] {
    \vspace{2pt}

	\parbox{\mpwidth}{
		\begin{tabular*}{1\mpwidth}{p{0.65\mpwidth}  r}
            {\Large \textcolor{black}{\textbf{#2}}} & \colorbox{maincol}{\makebox[0.33\mpwidth]{\textcolor{white}{\textbf{#1}}}} \\[3pt]
            {\large \textcolor{darkcol}{\textbf{#3}}} & \\[3pt]
		\end{tabular*}\\
	}
    \begingroup
        \setlength{\baselineskip}{15pt}
        \renewcommand{\CJKglue}{\hskip 1.5pt}
        #4
    \endgroup
}

\newcommand{\work}[2] {

        \cvtext{\textbf{#1}}

        \cvlist{#2}

}

\newcommand{\intershipUptoday}[4] {
    \vspace{2pt}

	\parbox{\mpwidth}{
		\begin{tabular*}{1\mpwidth}{p{0.69\mpwidth}  r}
            {\Large \textcolor{black}{\textbf{#2}}} & \colorbox{maincol}{\makebox[0.25\mpwidth]{\textcolor{white}{\textbf{#1}}}} \\[3pt]
            {\large \textcolor{darkcol}{\textbf{#3}}} & \\[3pt]
		\end{tabular*}\\
	}
    \begingroup
        \setlength{\baselineskip}{15pt}
        \renewcommand{\CJKglue}{\hskip 1.5pt}
        #4
    \endgroup
}


%----------------------------------------------------------------------------------------
%	main
%----------------------------------------------------------------------------------------
\begin{document}

\columnratio{0.75} % 左栏占比75%
\setlength{\columnsep}{2.2em} % 左右栏间距
\setlength{\columnseprule}{4pt} % 左右栏间距宽度
\colseprulecolor{lightcol} % 左右栏间距颜色

% 分栏
\begin{paracol}{2}
    % 左栏
    \begin{leftcolumn}

        \mysection{教育经历}
        \education
        {2020-09\quad 至今}
        {浙江理工大学}
        {计算机科学与技术\enspace (全英文)\enspace |\enspace 本科}
        {GPA\enspace 3.76/5.00\enspace (前15 \%)}
        {相关课程}
        {
            \vspace{-0.5\baselineskip}
            \begingroup
            \begin{multicols}{2}
                \setlength{\columnseprule}{0pt}
                \begin{itemize}
                    \item C程序设计 { \hfill (96) \qquad }

                    \item 计算机网络 { \hfill (95) \qquad }

                    \item 数据结构与算法 { \hfill (96) \qquad }

                    \item 面向对象程序设计 { \hfill (93) \qquad }

                \end{itemize}
                \vspace{-\baselineskip}
            \end{multicols}
            \vspace{-\baselineskip}
            \endgroup
            \vspace{-\baselineskip}
        }

        \vspace{-20pt}

        \vfill\null

        \mysection{实习经历}

        \intership
        {2023-04\enspace 至\enspace  2023-07}
        {杭州登虹科技有限公司}
        {测试开发}
        {
            \work{工作职责}{
                \item \href{https://www.closeli.cn/}{\textbf{\emph{登虹科技}}主要的业务在图像处理、计算机视觉和云计算等领域。作为测试开发实习生,我主要负责IHP部门的稳定性测试、压力测试、光敏测试、以及部分系统测试项目,负责三条云智能监控摄像头产品线的测试,及测试过程中的开发相关工作}
                }


            \work{完成工作}{
                \item 在稳定性实验室负责三个产品线设备的测试工作。通过编写实验室设备管理程序,定时爬虫获取设备日志,修改测试规范与流程,将原有工作效率提升\textbf{300\%以上}
                \item 使用Airtest编写模块化自动化测试脚本,用于OTA升级压力测试,提高了自动化测试脚本在不同产品线之间的可移植性
                \item 测试过程中发现并协同多个部门解决一项重大bug。使用Tcpdump和WireShark进行网络抓包,
                }
        }
        
        \vspace{5pt}

        \intership
        {2023-04\enspace 至\enspace 2023-06}
        {杭州深软科技有限公司}
        {Python后端开发}
        {
            
            \work{工作职责}{
                \item 领导三人小组,在三个月内完成了\textbf{人脸识别系统}的设计,开发,测试,部署等工作,最终识别延迟在\textbf{1s以内},识别准确率在\textbf{95\%以上},最终应用至\textbf{浙江理工大学计算机学院各实验室}
                \item 123456
                }
        }

        \vspace{-20pt}

        \vfill\null

        \mysection{项目经历}
        \project
        {2022-03\enspace 至\enspace 2022-08}
        {基于深度学习的商品识别系统}
        {\cvlist{
                \item \textbf{项目介绍:}针对在无人货架中的应用,对常见的20余种商品,拍摄了共5422张不同角度,不同组合的图片,以及对应的标注文件作为数据集进行训练,最终实现对商品的识别
                \item \textbf{主要职责:}使用Python语言,\textbf{PyTorch}框架,\textbf{YOLOv5}算法。训练过程中对训练数据集使用了Mosaic拼接、图片覆盖混入、随机翻转等数据增强方式。通过200轮的迭代,最终使\textbf{mAP 0.5:0.95}达到\textbf{0.82}以上
                \item \textbf{项目收获:}项目最终部署至RKNN边缘智能计算平台以及硬件设备中。项目获得中国大学生计算机设计大赛\textbf{国家级三等奖}
            }
        }
        \vspace{5pt}

        \project
        {2022-09\enspace 至\enspace 2022-12}
        {智能知识侦察助手}
        {\cvlist{
                \item \textbf{项目介绍:}爬取豆瓣图书Top200的书籍,编写对图书信息侦查的问卷生成程序,以网页形式呈现出来,实现问卷填写,数据分析,可视化展示等功能
                \item \textbf{主要职责:}本人主要负责\textbf{后端开发}与\textbf{服务器部署}。后端使用Python语言,应用面向对象程序设计的思想抽象出书籍、问题、数据库连接、控制等多个类进行操作。最终部署至Linux服务器。网页链接:\href{http://castamerego.com/}{\textbf{\emph{castamerego.com}}}
                \item \textbf{项目收获:}三人小组使用Github,拥有\textbf{超过100条}commit记录,并有完整的过程记录。具体内容可访问\enspace \href{https://github.com/Castamere/Read-Book}{\textbf{\emph{github.com/Castamere/Read-Book}}}\enspace 查看,过程中培养了良好的多人协作coding能力。
            }
        }
        % \vspace{5pt}

        % \project
        % {2022-09\enspace 至\enspace 2022-12}
        % {}
        % {\cvlist{
        %         \item \textbf{项目介绍:}
        %         \item \textbf{主要职责:}
        %         \item \textbf{项目收获:}
        %     }
        % }
        \vspace{5pt}

        \project
        {2021-09\enspace 至\enspace 2021-10}
        {大众点评爬虫}
        {\cvlist{
                \item \textbf{数据爬取:}掌握Html/CSS/Javascript,爬取数据使用Selenium, Requests库。针对大众点评的css-svg映射反爬取进行逆向、使用代理ip防止被封
                \item \textbf{数据清洗:}使用BeautifulSoup4,re库。使用id,class,xpath等多种方式对元素定位
                \item \textbf{数据储存:}使用Pandas库储存至Excel表格,使用PyMySQL库储存至数据库
            }
        }
        \vspace{5pt}

        \project
        {2023-05\enspace 至\enspace 2023-06}
        {QBomb}
        {\cvlist{
                \item \textbf{项目介绍:}本项目是一款炸弹人风格的Android游戏,玩家可以放置炸弹来摧毁障碍物并杀死AI敌人,地图中还有各种道具来增强各种属性。 将吃豆人与泡泡堂两款经典游戏的亮点结合,有较高的可玩性
                \item \textbf{主要职责:}使用Android Studio,Java语言进行编程。主要负责编写游戏内玩家、道具、AI机器人、游戏判定等逻辑,以及游戏界面的设计
                \item \textbf{项目收获:}参与项目的前期整体设计与资源搜集,中期逻辑设计与编程实现,后期整体测试与文档撰写。各个流程都有一定的收获,可在\href{https://github.com/Casta-mere/QBomb}{\textbf{\emph{github.com/Casta-mere/QBomb}}}查看并下载APK体验
            }
        }
        \vspace{5pt}

        \project
        {2023-04\enspace 至\enspace 2023-06}
        {在线人脸识别系统}
        {\cvlist{
                \item \textbf{项目介绍:}本项目为一款实验室人脸识别考勤系统,前端为Web端,后端则基于flask开发框架,使用dlib的人脸识别框架
                \item \textbf{主要职责:}本人主要负责整体项目功能与结构设计,后端开发与服务器部署。设计并实现了\textbf{单服务器多客户端}的服务结构,通过多线程确保不会出现某个客户端意外离线导致服务器端出现堵塞,以及通过设置定时器防止僵尸进程等
                \item \textbf{项目收获:}领导三人小组,在三个月内完成了人脸识别系统的设计,开发,测试,部署等工作,最终识别延迟在\textbf{1s以内},识别准确率在\textbf{95\%以上},最终应用至\textbf{浙江理工大学计算机学院各实验室}
            }
        }

        \vspace{-20pt}

        \vfill\null

        \mysection{获奖经历}

        \cvlist{
        \item 中国大学生计算机设计大赛  { \hfill \textbf{国家级三等奖} }
        
        \item 挑战杯``揭榜挂帅''专项赛  { \hfill \textbf{国家级三等奖} }

        \item 中国大学生英语阅读大赛  { \hfill \textbf{校级特等奖} }

        \item ACM程序设计大赛新生赛  { \hfill 校级一等奖 }

        \item 大学生电子商务竞赛  { \hfill 省级三等奖 }

        \item 互联网+大学生创新创业大赛  { \hfill 校级二等奖 }

        \item 浙江省省政府奖学金  { \hfill \textbf{省政府奖学金} }

        \item 校二等奖学金、三等奖学金、优秀学生干部、社会工作奖学金、优秀团干部

        }
        \vspace{-20pt}

        \vfill\null

        \mysection{自我评价}

        \cvlist{
            \item 担任全英文班班长,经常帮助留学生处理问题,有较好的英文日常对话能力,做事认真负责
            \item 课本均为黑皮书系列英文原著,有较好的英文文献阅读能力
            \item 善于运用chatGPT加成的New bing以及Github copilot解决问题
            \item 有良好的coding风格与规范,有良好的面向对象程序设计思想
            \item 有丰富的Python开发经验,Python爬虫经验,熟悉常用库如Pymysql,selenium,requests,BeautifulSoup4,Pandas等
            \item 有丰富的社区代码提交经验,在Github上有130+次contribution
            \item 熟练掌握Linux常用命令,Git工具,有Linux服务器部署经验
        }

        \vfill\null

    \end{leftcolumn}

    % 右栏
    \begin{rightcolumn}

        {\begin{center}\includegraphics[width=0.8\linewidth]{蓝底渐变.png}\end{center}}

        \vfill\null

        \name{王旭刚}

        \icontext{Calendar}{12}{2002-4 \enspace 2024届}{black}\\[6pt]
        \icontext{MobilePhone}{12}{138\,-\,3423\,-\,0484}{black}\\[6pt]
        \icontext{Qq}{12}{2287245796}{black}\\[6pt]
        \iconemail{Envelope}{12}{castamerego@gmail.com}{castamerego@gmail.com}{black}\\[6pt]
        \iconhref{Github}{12}{github.com/Casta-mere}{https://github.com/Casta-mere}{black}\\[6pt]
        \icontext{MapMarker}{12}{浙江 \enspace 杭州}{black}
        % \iconhref{Globe}{12}{castamerego.com}{http://castamerego.com/}{black}\\[6pt]

        \vfill\null

        \mysection{专业技能}

        \skill{Python} {3+ yrs} {1} \\[-2pt]

        \skill{C++} {3+ yrs} {0.92} \\[-2pt]

        \skill{Android(Java)} {1+ yr} {0.86} \\[-2pt]

        \skill{Linux} {2+ yrs} {0.59} \\[-2pt]

        \skill{GIT} {3+ yrs} {0.68} \\[-2pt]

        \skill{Docker} {1+ yr} {0.5} \\[-2pt]

        \skill{MySQL} {2+ yrs} {0.68} \\[-2pt]

        \skill{\LaTeX } {2+ yrs} {0.82} \\[-2pt]

        \skill{HTML/CSS/JS} {掌握} {0.46} \\[-2pt]

        \skill{Ms Of\mbox{}f\mbox{}ice} {精通} {1} \\[-2pt]

        \vfill\null

        \mysection{资格证书}

        \certification{CET-4}{\texttt{571}}

        \certification{CET-6}{\texttt{598}}

        \vfill\null

        \mysection{最近阅读}

        \cvtext{Python-cookbook}

        \cvtext{《动手学深度学习》 李沐}

        \vfill\null

        \mysection{工具}

        \cvtext{SSH}

        \cvtext{SecureCRT}

        \cvtext{MobaXterm}

        \cvtext{Xmind}
        
        \cvtext{Notion}
        
        \cvtext{WireShark}
        
        \cvtext{Visio}
        
        \cvtext{GitKarken}

        \vfill\null

    \end{rightcolumn}

\end{paracol}

\end{document}